\documentclass[11pt]{article}

\usepackage{cmap}

\usepackage[utf8]{inputenc}
\usepackage[T1]{fontenc}
\usepackage[english]{babel}

\usepackage{microtype}
\usepackage{csquotes}
\usepackage{enumitem}
\usepackage{amsmath}
\usepackage{amsthm}
\usepackage{amsfonts}
\usepackage{mathtools}

\setlength{\parindent}{0cm}

\newtheorem{definition}{Definition}
\newtheorem{example}{Example}

\begin{document}

\section{Adverse Selection}

In reality, information is often asymmetrically held by market participants. Examples of such situations are

\begin{itemize}
	\item Firms do not know the abilities of workers to be hired
	\item Investors / creditors cannot observe the abilities / projects of firms' managers
	\item Automobile insurance companies do not know the individual driving skills of their customers
	\item Buyers of used car can rarely check the cars' qualities perfectly
\end{itemize}

Such hidden information can lead to adverse selection, signalling and screening. Following, we will examine the case of adverse selection.

\subsection{Basic Labour Market Model}

Assume that in a market, there are many workers of different types $\theta$, e.g. with different productivity, such that

	$$ \theta \in [\underline{\theta}, \overline{\theta}] \text{ with }0 \leq \underline{\theta} < \overline{\theta} < \infty.  $$
	
The proportion of workers with productivity of $\theta$ or less is given by the cumulative distribution function $F(\theta)$. Additionally, we assume there to be many identical potential firms that can hire workers seeking to maximise their expected profits. A worker can choose to work either at a firm or at home, and the utility of worker $\theta$ is thereby given by

	$$ u(w, \theta) = \begin{cases} \omega & \text{ if he accepts wage } \omega \\ r(\theta) & \text{ if he does not accept } \omega \end{cases} $$
	
Firms produce with labour only and have constant production technologies

	$$ \theta = \text{ output (=revenue) of worker } \theta $$
	
Pareto-optimum: Set of workers employed is $\theta = \left\{ \theta \colon \theta \geq r(\theta) \right\}$.

\subsubsection{Case 1: $\theta$ is publicly observable}

The resulting wage will depend on each type, i.e. $\omega(\theta)$. Because the labour of each different type of worker is a distinct, publicly known good, there is a distinct equilibrium wage. As the profit of a firm ($\pi$) from a type $\theta$ worker:

	$$ \pi = \theta - \omega(\theta) $$

They will only seek employment if $\theta > \omega(\theta)$. The resulting demand for labour ($z$) of type $\theta$ is hence:

	$$ z(\theta) = \begin{cases} 0 & \text{ if } \theta < \omega(\theta) \\ [0, \infty) & \text{ if } \theta = w(\theta) \\ \infty & \text{ if } \theta > \omega(\theta) \end{cases} $$

Therefore, given the competitive, constant returns nature of the firm, in a \textit{competitive} equilibrium the different wage offers (i.e. $\pi = 0$) are
 
	$$ \omega(\theta) = \theta, $$
	
and the set of workers accepting employment in a firm is

	$$ \Theta = \left\{ \theta \colon r(\theta) \leq \theta \right\} $$
	
As expected from the first fundamental welfare theorem, the outcome is, due to the competitive nature of the market, a Pareto optimal equilibrium. Note: For this result firms need not know $r(\theta)$, and thus, the market can successfully aggregate and process information!

\subsubsection{Case 2: $\theta$ is ex ante observable only to the worker}

We have to begin by noting that the resulting equilibrium in this competitive environment is of asymmetric information. A firm will learn about $\theta$ only \textit{ex post}, therefore  this is not useful for \textit{ex ante} contracting. This means the wage cannot depend on $\theta$, and hence, there must be a single wage $\omega$ offered to all workers. Again, a worker of type $\theta$ is willing to work for a firm if and only if $r(\theta) \leq \omega$. Hence, the set of worker types who are willing to accept employment at wage rate $\omega$ is

	$$ \Theta(\omega) = \left\{ \theta \colon r(\theta) \leq \omega \right\} $$
	
and gives us the labour supply. Let $\mu$ be the expected average productivity of employed workers (to be determined). If a firm assumes $\mu$, its demand for labour is given by

	$$ z(\omega)= \begin{cases} 0 & \text{ if } \mu < \omega \\ [0, \infty) & \text{ if } \mu = \omega \\ \infty & \text{ if } \mu > \omega \end{cases} $$ 
	
Now, if worker types in set $\Theta^*$ are accepting employment offers in a competitive equilibrium, and if firms' beliefs about the productivity of potential employees correctly reflect the actual average productivity of the workers hired in this equilibrium, then we must have 

	$$ \mu = \mathbb{E} \left[\theta ~|~\theta \in \Theta^* \right] $$
	
	Hence, the labour demand implies that the demand can equal its supply in an equilibrium with a positiv level of employment if and only if 
	
	$$ w = \mathbb{E} \left[ \theta ~|~ \theta \in \Theta^* \right] $$
	
	This leads to the notion of a competitive equilibria
	
	\begin{definition}
		A \textbf{competitive equilibrium} is a wage $\omega^*$, an employment set $\Theta^*$, and a belief $\mu^*$ such that
		
		\begin{align*}
			\Theta^* & = \left\{ \theta \colon r(\theta) \leq \omega^* \right\} \\
			\omega^* & = \mu^* \\
			\mu^* & = \begin{cases} \mathbb{E} \left[ \theta ~|~\theta \in \Theta^* \right] & \text{ if } \Theta^* \neq \emptyset \\ \mathbb{E} \left[ \theta \right] & \text{ if } \Theta^* = \emptyset \end{cases}
		\end{align*}
	\end{definition}

Note, that the expectation in the definition above seems not to be well defined when no workers are accepting employment in an equilibrium, i.e. when $\Theta^* = \emptyset$. In the following discussions however, we can assume for simplicity that in this circumstance each firm's expectation of potential employees' average productivity is simply the unconditional expectation $\mathbb{E} \left[ \theta \right]$, and we take 

	$$ \omega^* = \mathbb{E} \left[ \theta \right] $$
	
in any such equilibrium. Typically a competitive equilibrium as defined above will fail to be Pareto optimal which is seen in the following example.

\begin{example}
	Assume $r(\theta) = r$ for all $\theta$, with $\underline{\theta} < r < \overline{\theta}$, i.e. every worker is equally productive at home. The Pareto optimal allocation of labour in this setting has workers with $\theta \geq r$ accepting employment at a firm and those with $\theta < r$ not doing so. In the competitive equilibrium, a worker is willing to accept employment if $\omega \geq r$. This means at a given wage, $\Theta(\omega)$, is either:
	
	$$ \Theta = \left[ \underline{\theta}, \overline{\theta} \right] \text{ if } \omega \geq r ~\text{ or }~ \Theta = \emptyset \text{ if } \omega < r $$
	
	Thus, in either case, $\mu = \mathbb{E}\left[ \theta ~|~\theta \in \Theta^*(\omega) \right] = \mathbb{E} \left[ \theta \right]$, i.e. for all $\omega$ and unconditional. By the definition of competitive equilibria  the equilibrium wage rate must be
	
		$$ \omega^* = \mathbb{E} \left[ \theta \right] $$
		
	If $\mathbb{E}[\theta] \geq r$, then all workers accept employment at a firm; if $\mathbb{E}[\theta] < r$, then none do. Which type of equilibrium arises depends on the relative fraction of good and bad workers. For example, if there is a high fraction of low-productivity workers then, because firms cannot distinguish good workers from bad, they will be unwilling to hire any workers at a wage that is sufficient to have them accept employment, i.e.
	
		$$ \mathbb{E} \left[ \theta \right] < r $$
		
	On the other hand, if there are very few low-productivity workers, then the average productivity of the workforce will be avoce $r$, and so the firms will be willing to hire workers at a wage that they are willing to accept, i.e.
		
	$$ \mathbb{E} \left[ \theta \right] \geq r $$
	
	In one case, too many workers are employed relative to the Pareto optimal allocation, and in the other too few, and therefore,  neither case is Pareto-optimal since 
	
	$$\Theta^* \neq \left\{ \theta \colon \theta \geq r \right\}. $$
	
	The cause of this failure of the competitive allocation to be Pareto optimal is simple to see: because firms are unable to distinguish among workers of differing productivities, the market is unable to allocate workers efficiently between firms and home production. ~\bigskip
	
	Note: If workers could not observe their own type $\theta$ the same allocation would result. Thus, knowledge does not do harm in this example, but cannot be optimally processed by the market.
\end{example}

\section{Signalling}

In the model with adverse selection, good workers were underpaid because they were unable to credibly inform firms about their productivity. Hence, they should have an incentive to \enquote{signal} their productivity, if this is possible. This will now be investigated in a simple setting.

\subsection{Simple Signalling}

\subsubsection{Model}

In this section we assume that there are two firms competing for hiring workers. Furthermore, there are only two types of workers, $L$ and $H$, with $0 < \theta_L < \theta_H$ and $\lambda = \mathbb{P}(\theta = \theta_H)$ where $0 < \lambda < 1$. We assume 

	$$ 0 \leq r(\theta_L) \leq r(\theta_H) < \mathbb{E}[\theta]. $$

The important extension of our previous model is that before entering the job market a worker can signal, this can have the form of getting some education and the amount of education that a worker receives is observable. Regardless of the type, signalling does nothing for a worker's productivity. Type $\theta_H$  can produce such a signal at cost $c$, and we assume that type $\theta_L$ cannot produce a signal, where

	$$ 0 \leq c < \theta_H - \mathbb{E}[\theta] $$

\subsubsection{Timing}

In this setting, we can identify four stages developing from setup to possible employment:

\begin{enumerate}
	\item Nature chooses workers' types, observed only by them
	\item Workers may produce a signal
	\item Firms observe signals and simultaneously announce wage offers
	\item Workers decide whether to join a firm or not. If they are indifferent, they choose either option with probability $p = 0.5$.
\end{enumerate}

\textbf{Stage 3}: ~\smallskip

At the fourth stage the decision is merely where a worker maximises his profit, we will therefore start with stage three. Here, we notice that wages cannot depend on the type $\theta$ as those aren't observable. However, the wages can depend on the signal:

\begin{itemize}
	\item Let $\omega_s$ be the wage offered to a worker who has signalled
	\item Let $\omega_{nonS}$ be the wage offered to a worker who has not signalled.
\end{itemize}

The competitive equilibrium that emerges is hence
$$ \omega_s = \theta_H ~\text{ and }~ \omega_{nonS} \in \left[\theta_L, \mathbb{E}[\theta] \right] $$

\textbf{Stage 2}: ~\smallskip

As by assumption $0 \leq c < \theta_H - \mathbb{E}[\theta]$ holds, all workers of type $H$ will now signal.

\subsubsection{Consider two sub-cases}

\begin{itemize}
	\item $r(\theta_L) \leq r(\theta_H) \leq \theta_L$: Signalling is wasteful
	\item $\theta_L < r(\theta_L) < r(\theta_H) < \mathbb{E} \left[ \theta \right]$: Signalling may be useful
\end{itemize}

Hint: In both cases, the equilibrium in the pure adverse selection model (without signalling) is 

	$$ \omega^* = \mathbb{E}[\theta] $$

\end{document}